\documentclass[11pt]{article}
\usepackage[margin=1.0in]{geometry}

\begin{document}
\title{Maxwell's Demon Algorithm Human Subject Protocol}
\author{}
\date{\today}
\maketitle
\section{Experiment Organization}

\	\	There will be 10 subjects total. Each subject will be tested under 4 conditions for 10 trials each. Half of the subjects will be randomly selected to begin with the Omni+MDA condition, while the other half will begin with the Omni without MDA. After the first 10 trials, the subjects will be switched to a more difficult system with faster dynamics, continuing with or without MDA based on which group they were randomly placed in. The remaining 3 conditions will be tested in random order. Each subject will begin with a short training session described in detail below, where they will be briefly introduced to all 4 conditions. Subjects will be allowed to explore the test systems and ask questions.

\section{Protocol for Individual subjects}
\begin{enumerate}
\item Go over consent form with subject, explaining study protocol and goals as outlined on the form.
\item Demonstration and introduction to software only MDA on phone.
	\begin{enumerate}
	\item[\textbf{No MDA:}] The phone approximates and acceleration based on the position of the finger on the touchscreen. This is used to drive the cart-pendulum simulation.
	\item[\textbf{MDA:}] The phone uses the same acceleration approxiation, but filters this input signal. The algorithm will either accept, reject, or saturate this input signal based on the output and saturation limits of the Sequential Action Control(SAC) algorithm. 
	\end{enumerate}
\item Demonstration and introduction to the mechanical MDA on Phantom Omni Haptic device(Sensable Technologies). Subjects will be asked to hold the stylus in a firm cylindrical grasp.
	\begin{enumerate}
	\item[\textbf{No MDA:}] The left-right position of the stylus is in put to a trep model, which uses a variational integrator to step the system forward in time. The position of the stylus is represented by a box marker in rviz. The position of the pendulum is calculated using trep and is also visualized in rviz. There is \textbf{no force rendering}.
	\item[\textbf{MDA:}] The simulation and visualization is update in the same way as when the MDA is not running. However, the user will experience two types of feedback force from the Omni, such that the algorithm will mechanically either accept, reject, or saturate this input signal based on the output and saturation limits of the Sequential Action Control(SAC) algorithm. 
		\begin{enumerate}
			\item \begin{enumerate}
			\item[\textbf{Saturation:}] Saturation is experienced as a drag force. The force will increase   if the stylus is moved to quickly or it is approaching the saturation velocity. It is designed to feel as though the stylus is being moved through a heavy fluid.
			\item[\textbf{Rejection:}] This type of feedback is designed to feel like a wall which is changing based on the SAC signal. If the stylus is moved in the direction opposite of the direction that SAC has prescribed, the algorithm will put up a 'wall'. When the wall is hit a very high force is generated by the Omni feeling like a sudden jolt.
			\end{enumerate}
		\end{enumerate}		  
	\end{enumerate}
	
\item Begin with the Omni, with or without MDA according to a random selection. To start the ROS package:\\
\texttt{\$ roslaunch trep\_omni\_cart cart\_pend.launch vis:=true replay:=false} \item Set up recording of the ROS topics by navigating to the \texttt{trep\_omni\_cart} directory and executing the command:\\ \texttt{\$ ./record.sh data/\textsl{\textsf{subjectcode}}-trial\textsl{\textsf{trialnumber}}}
Follow on screen instruction to start and stop recording. 
\item Each trial will be stopped automatically by the software when $$-0.15<\theta<0.15\ and\ -0.6<\dot{\theta}<0.6$$ or after 50 seconds have elapsed.

\item The time to task completion and SAC score will be recorded at this time.
\item Repeat steps 5-7 for for 10 trials.
\item After 10 trials, in the \texttt{trep\_omni\_cart} directory, switch to the higher difficulty system:\\
\texttt{\$ git checkout Esys}\\
and complete an additional 5 trials.
\item Complete the remaining 3 sets of trials in random order, following steps 4-7 for the remaining Omni tests and 6-7 for the touchscreen tests.

\end{enumerate}



\end{document}